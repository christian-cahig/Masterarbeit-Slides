\subsection{Warm-starting an OPF solver}

\begin{frame}[t]{APF for OPF: Providing solvers with warm-start points}{}
    OPF solvers are iterative:
    \textcolor<1>{CornellRed}{\(\OPFVars[k+1] \gets \operatorname{update}\!\OPFVars[k]\)}
    \begin{itemize}
        \item User-specified \textcolor<1>{CornellRed}{starting point \(\OPFVars[0]\)}
        \item \textcolor<1>{CornellRed}{Interior-point methods} are SOTA \cite{Capitanescu+2007},
            especially in large scale \cite{CapitanescuWehenkel2013,Kardos+2020v4,Kardos+2022}
    \end{itemize}

    \onslide*<2->{
    \vspace{1em}
    \textcolor<2>{CornellRed}{Snapshot-starting} a solver
    \begin{itemize}
        \item \(\OPFVars[0] \gets \SnapPoint\)
        \item[\ding{43}] Search \textcolor<2>{CornellRed}{does not start on PFM}
    \end{itemize}
    }

    \onslide*<3->{
    \vspace{1em}
    \textcolor<3>{CornellRed}{Warm-starting} a solver with APF point \(\APFPoint\)
    \begin{itemize}
        \item \(\OPFVar{c}{0} \gets \left(\SupPs + \PFESlack\PFESDist, \SupQs\right)\)
        \item \(\OPFVar{s}{0} \gets \BusMAs\)
        \item[\ding{43}] Search \textcolor<3>{CornellRed}{starts on PFM}
    \end{itemize}
    }
\end{frame}

\begin{frame}[t]{APF for OPF: Providing solvers with warm-start points}{}
    \begin{columns}[T]
    \begin{column}{0.5\textwidth}
    \onslide*<1->{
    Solve instances on \texttt{POL3375} via MIPS 1.4 \cite{MIPSv1.4}
    \begin{itemize}
        \item 408 APF instances

        \item Get snapshot- \& warm-started optima
        \begin{itemize}
            \item \(\Optim[s]{\SupInjs}\), \(\Optim[s]{\BusMs}\), \(\Optim[s]{\BranchAs}\), \(\Optim[s]{g}\)
            \item \(\Optim[w]{\SupInjs}\), \(\Optim[w]{\BusMs}\), \(\Optim[w]{\BranchAs}\), \(\Optim[w]{g}\)
        \end{itemize}

        \item Compare solutions in terms of
        \begin{itemize}
            \item \(\epsilon_{\mathsf{c}} \coloneqq \Norm*{\begin{matrix*}
                \Optim[s]{\SupInjs}-\Optim[w]{\SupInjs}\end{matrix*}}_{\infty}\)
                (in 100-MVA units)
            \item \(\epsilon_{\mathsf{v}} \coloneqq \Norm*{\begin{matrix*}
                \Optim[s]{\BusMs}-\Optim[w]{\BusMs}\end{matrix*}}_{\infty}\)
                (in 400-kV units)
            \item \(\epsilon_{\mathsf{a}} \coloneqq \Norm*{\begin{matrix*}
                \Optim[s]{\BranchAs}-\Optim[w]{\BranchAs}\end{matrix*}}_{\infty}\)
                (in radians)
            \item \(\epsilon_{\mathsf{g}} \coloneqq \Optim[s]{g} - \Optim[w]{g}\)
                (in USD)
        \end{itemize}

        \item[\ding{43}] <7->
            \textcolor<7->{CornellRed}{
            \(\left(\Optim[w]{g},\Optim[w]{\BusMs},\Optim[w]{\BranchAs}\right)
            \approxeq
            \left(\Optim[s]{g},\Optim[s]{\BusMs},\Optim[s]{\BranchAs}\right)\)}
            attained from \textcolor<7->{CornellRed}{
            distinct \(\Optim[w]{\SupInjs}\) and \(\Optim[s]{\SupInjs}\)} 
    \end{itemize}
    }
    \end{column}

    \begin{column}{0.5\textwidth}
    \onslide*<2-7>{
    \hspace*{\fill}{\small At \(\XEDRegs=\left(0,0\right)\)}
    \begin{table}[t!] \vspace{-0.7em} \small
    \begin{tabularx}{\textwidth}{@{} cCC @{}}
    \toprule
    Metric & Minimum & Maximum
    \\ \midrule
    \(\epsilon_{\mathsf{g}}\)
    & \textcolor<3>{BrandeisBlue}{\(-1.7847\times10^{-2}\)}
    & \textcolor<3>{BrandeisBlue}{\(1.90488\times10^{-2}\)}
    \\
    \(\epsilon_{\mathsf{v}}\)
    & \textcolor<4>{LightSalmon}{\(4.79369\times10^{-9}\)}
    & \textcolor<4>{LightSalmon}{\(6.89567\times10^{-5}\)}
    \\
    \(\epsilon_{\mathsf{a}}\)
    & \textcolor<4>{LightSalmon}{\(6.47442\times10^{-10}\)}
    & \textcolor<4>{LightSalmon}{\(5.92979\times10^{-6}\)}
    \\
    \(\epsilon_{\mathsf{c}}\)
    & \textcolor<5-7>{CornellRed}{\(6.05418\times10^{-4}\)}
    & \textcolor<5-7>{CornellRed}{\(8.78352\times10^{-1}\)}
    \\
    \bottomrule
    \end{tabularx}
    \end{table}
    }

    \onslide*<6-7>{
    \hspace*{\fill}{\small At \(\XEDRegs=\left(1,1\right)\)}
    \begin{table}[t!] \vspace{-0.7em} \small
    \begin{tabularx}{\textwidth}{@{} cCC @{}}
    \toprule
    Metric & Minimum & Maximum
    \\ \midrule
    \(\epsilon_{\mathsf{g}}\)
    & \(-1.21094\times10^{-2}\)
    & \(2.14678\times10^{-2}\)
    \\
    \(\epsilon_{\mathsf{v}}\)
    & \(1.82573\times10^{-9}\)
    & \(2.4873\times10^{-5}\)
    \\
    \(\epsilon_{\mathsf{a}}\)
    & \(2.88286\times10^{-10}\)
    & \(3.47572\times10^{-6}\)
    \\
    \(\epsilon_{\mathsf{c}}\)
    & \textcolor<6-7>{CornellRed}{\(1.23703\times10^{-4}\)}
    & \textcolor<6-7>{CornellRed}{\(2.88003\times10^{-1}\)}
    \\
    \bottomrule
    \end{tabularx}
    \end{table}
    }

    \onslide*<8>{
    \hspace*{\fill}{\small At \(\XEDRegs=\left(1,0\right)\)}
    \begin{table}[t!] \vspace{-0.7em} \small
    \begin{tabularx}{\textwidth}{@{} cCC @{}}
    \toprule
    Metric & Minimum & Maximum
    \\ \midrule
    \(\epsilon_{\mathsf{g}}\)
    & \textcolor<8>{BrandeisBlue}{\(-2.44578\times10^{-2}\)}
    & \textcolor<8>{BrandeisBlue}{\(2.24346\times10^{-2}\)}
    \\
    \(\epsilon_{\mathsf{v}}\)
    & \textcolor<8>{LightSalmon}{\(7.57086\times10^{-10}\)}
    & \textcolor<8>{LightSalmon}{\(6.40276\times10^{-5}\)}
    \\
    \(\epsilon_{\mathsf{a}}\)
    & \textcolor<8>{LightSalmon}{\(1.99231\times10^{-10}\)}
    & \textcolor<8>{LightSalmon}{\(4.99960\times10^{-6}\)}
    \\
    \(\epsilon_{\mathsf{c}}\)
    & \textcolor<8>{CornellRed}{\(6.91804\times10^{-4}\)}
    & \textcolor<8>{CornellRed}{\(1.16527\times10^{0}\)}
    \\
    \bottomrule
    \end{tabularx}
    \end{table}
    }
    \end{column}
    \end{columns}
\end{frame}

\begin{frame}[t]{APF for OPF: Providing solvers with warm-start points}{}
    \begin{block}<1->{It's just the nonconvexity of OPF}
        The APF point
        \textcolor<1->{CornellRed}{can be sufficiently far} from the snapshot point
        that, for \textcolor<1->{CornellRed}{the same algorithm},
        these starting points lead to \textcolor<1->{CornellRed}{distinct optima}.
    \end{block}

    \begin{corollary}<2->{}
        In its current form, APF is
        \textcolor<2->{CornellRed}{a crude method for finding multiple OPF solutions}.
    \end{corollary}

    \begin{alertblock}<3->{Open problems}
    \begin{enumerate}
        \item Given a solver, a snapshot point \(S\), and a APF point \(A\),
            how can we tell that starting the solver at \(S\) and at \(A\)
            will or will not give us distinct optima?
        \item How to make APF a more disciplined method of finding multiple OPF solutions?
    \end{enumerate}
    \end{alertblock}
\end{frame}
