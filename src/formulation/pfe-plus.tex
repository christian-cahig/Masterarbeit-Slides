\subsection{Solving for the anticipated bus voltages}

\begin{frame}[t]{Solving for the anticipated bus voltages}{}
    \begin{block}{APF equations (\APFE)}
    \vspace{-1em}
    \begin{IEEEeqnarray*}{rCcCl}
        \Resids\paren{\PFEVars;\Ybus,\DemDrws,\SupInjs,\BusA{\PFERefBus}}
        &\coloneqq&
        \underbrace{
            \NetInjs \paren{\BusMs,\PFEBusAs;\Ybus,\BusA{\PFERefBus}}
            \ - \
            \PFESlack \begin{bmatrix*}\SupCon\PFESDist \\ \ZerosVec[\NumBuses]\end{bmatrix*}
        }_{\PFEResids\paren{\PFEVars;\,\Ybus,\,\BusA{\PFERefBus}}}
        -
        \begin{bmatrix*}\SupCon\SupPs \\ \SupCon\SupQs\end{bmatrix*}
        -
        \begin{bmatrix*}\DemCon\DemPs \\ \DemCon\DemQs\end{bmatrix*}
        &=&
        \ZerosVec[2\NumBuses]
    \end{IEEEeqnarray*}
    \end{block}

    \begin{itemize}
        \item <1-> A \textcolor<1>{CornellRed}{\(2\NumBuses\)--dimensional root-finding} task
        \begin{itemize}
            \item[\ding{43}] <1-> Lots of derivative-based algorithms with fast convergence guarantees
            \item <1-> See \S 3.3.2 for the \textcolor<1>{CornellRed}{APF Jacobian
                \(\APFJac\paren{\PFEVars} \,\coloneqq
                \partial_{\PFEVars}\PFEResids\paren{\PFEVars}\)}
            % \item <1-> See \S 3.3.3 for an initialization strategy
        \end{itemize}

        \item <2> \textcolor<2>{CornellRed}{\(\SupInjs\) is a parameter of \APFE}
        \begin{itemize}
            \item[\ding{43}] <2> Different \(\SupInjs\)'s give different \(\PFEVars\)'s
            \item[\ding{43}] <2> Compare APF points by their \(\PFESlack\)'s
        \end{itemize}
    \end{itemize}
\end{frame}
