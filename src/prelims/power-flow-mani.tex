\subsection{Power flow manifold}

\begin{frame}[t]{Power flow manifold (PFM)}{%
    A geometric intuition for the PFE}

    \begin{block}{PFE as a manifold in \(\left(\SupInjs,\BusVolts\right)\)--space}
        Expressed as
        {\color{CornellRed}
        \(\Resids\paren{\SupInjs,\BusVolts;\Ybus,\DemDrws} \ = \ZerosVec[2\NumBuses]\)},
        the PFE describe a manifold of all points \(\left(\SupInjs,\BusVolts\right)\)
        that are power-flow feasible for \(\left(\Ybus,\DemDrws\right)\).
    \end{block}

    \begin{itemize}
        \item PFE parameters \(\left(\Ybus,\DemDrws\right)\)
            dictate the \textcolor<2>{CornellRed}{``shape'' of PFM}

        \item Power-flow feasible \(\left(\SupInjs,\BusVolts\right)\)
            for \(\left(\Ybus,\DemDrws\right)\)
            \(\Longrightarrow\)
            \textcolor<3>{CornellRed}{\(\left(\SupInjs,\BusVolts\right)\)
            is on PFM for \(\left(\Ybus,\DemDrws\right)\)}

        \item Computation over PFE for \(\left(\Ybus,\DemDrws\right)\)
            \(\Longrightarrow\)
            \textcolor<3>{CornellRed}{finding a point on PFM for \(\left(\Ybus,\DemDrws\right)\)}
    \end{itemize}

    \vspace{5em}
\end{frame}
