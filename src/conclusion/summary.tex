\begin{frame}[t]{APF in summary}{}
    \begin{itemize}
        \item \textcolor<1>{CornellRed}{Formulation}:
            finding power-flow feasible \(\left(\SupInjs,\BusVolts\right)\)
            for anticipated grid conditions \(\left(\Ybus,\DemDrws\right)\),
            using preceding snapshot values \(\SnapPoint\)
        \begin{itemize}
            \item Anticipate \(\SupInjs\) by solving a convex program
                \textcolor<1>{CornellRed}{\XED[full]} (\XED)
            \item Compute corresponding \(\BusVolts\) by solving
                \textcolor<1>{CornellRed}{\APFE[full]} (\APFE)
        \end{itemize}

        \vspace{0.5em}
        \item \textcolor<2>{CornellRed}{Computation}:
            amply handled by existing and readily available tools
        \begin{itemize}
            \item SeDuMi and SDPT3 for \XED;
                Levenberg-Marquardt and Powell hybrid for \APFE
            \item \textcolor<2>{CornellRed}{Sub-second run times}
                on 3374-bus, 4161-branch, 596-generator portion of Polish grid
            \item \textcolor<2>{CornellRed}{Quadrimodal effect} of \XED
                \(\Longrightarrow\)
                \textcolor<2>{CornellRed}{big-\(\mu\) trick} for easily finding four APF points
        \end{itemize}

        \vspace{0.5em}
        \item \textcolor<3>{CornellRed}{Applications}
        \begin{enumerate}
            \item Warm-starting OPF solvers
                \(\Longrightarrow\)
                \textcolor<3>{CornellRed}{a crude method for finding multiple OPF solutions}

            \item Differentiating through \APFE
                \(\Longrightarrow\)
                \textcolor<3>{CornellRed}{power flow equations as a layer in amortized OPF}
        \end{enumerate}
    \end{itemize}
\end{frame}
